\section{Problemanalyse}   %wo liegen die Probleme oder Schwierigkeiten in einer Ligrettosimulation mit verteilten Systemen?

%allgemein, was muss analysiert werden

Als Aufgabe gilt es den durch die Spielregeln festgelegten Ablauf anhand folgender Gesichtspunkte in einen Algorithmus umzusetzen:

\begin{enumerate}
	\item Der Spielablauf muss den Spielregeln entsprechen und die KI-Algorithmen, welche die Mitspieler simulieren, sollten nur an Informationen über den Spielablauf gelangen, welche auch ein menschlicher Spieler gelangen würde.
	\item Der Spielablauf sollte möglichst fair sein. D.h. bis auf den Zufall, der durch das Mischen der Karten ins Spiel gelangt, sollten alle Mitspieler gleichgestellt sein.
	\item Die verwendeten Algorithmen und Schnittstellen sollten möglichst gut mit der Anzahl Mitspieler skalieren, unter Betrachtung der benötigten Rechenzeit und Netzwerkkapazität.
\end{enumerate}

Folgende Punkte werden bei der Herleitung des Algorithmus nicht betrachtet:

\begin{enumerate}
	\item Alle Mitspieler müssen sich an die Spielregeln halten, insbesondere in Teilen des Ablaufs, die im Spiel mit menschlichen Spielern nicht überwacht werden können. Dazu gehört z.B. auch das regelkonforme Ablegen der Handkarten beim suchen einer passenden Karte.
	\item Die beiligten Geräte versenden nur Nachrichten an Geräte, mit denen sie laut dem Alogrithmus kommunizieren dürfen.
	\item Die eingesetzte Soft- und Hardware darf keine Fehler aufweisen und der Netzerkverkehr darf nicht gestört oder unterbrochen werden (z.B. dürfen keine Pakete verloren gehen).
\end{enumerate}

%Für die letzten beiden Punkte wird anschliessend eine Diskusion gegeben, inwiefern der Alogorithmus und die Schnittstellen angepasst werden müssten, um ...

\subsection{Systemgrenzen}

%welche teile des algorithmus laufen wo

Damit die Laufzeit eines Spiels nicht (oder nur schwach) mit der Anzahl Spieler wächst, muss die Anzahl verwendeter Computer linear mit der Anzahl Mitspieler wachsen können. Deshalb ist es notwendig und sinnvoll, für jeden Mitspieler mindestens einen separaten Prozess oder Thread zu verwenden.

Da ein einzelner Spieler jedoch zu jeder Zeit immer nur eine Aktion ausführen darf (gemäss Spielregeln), ist eine Aufteilung eines einzelenen Spielers in mehrere Prozesse oder Threads nicht sinnvoll.

Zur Überwachung des gesammten Spiels sowie dessen Aufbau und Abbau wird ein separater Prozess verwendet. Da der überwachende Prozess nur vor Spielbeginn und nach Beendigung des Spiels eine Aufgabe hat, dessen Komplexität mit der Anzahl Mitspieler wächst, ist eine Aufteilung dieser Aufgaben in mehrere Prozesse oder Threads zwar erwägenswert, aber nicht von oberster Priorität.

% TODO: Nomenklatur für den Überwachende Prozess und Mitspieler-Prozesse

\newpage

Zusammengefasst:

\begin{enumerate}
	\item Ein Prozess zur Überwachung des Spiels (Aufbau und Abbau).
	\item Ein Prozess pro Mitspieler, welche alle Aktionen des jeweiligen Mitspielers durchführt.
\end{enumerate}


\subsection{Verteilung}

%welcher state ist wo

Bei der Betrachtung der Verteilung des Zustandes gibt es drei Arten von Objekten:

\begin{enumerate}
	\item {\bf Zustands-Objekte} verkörpern den veränderlichen Zustand eines Spiels. Die Eigenschaften dieser Objekte sind somit mutierbar. Für diese Objekte ist immer genau ein Prozess zuständig und diese Objekte werden auf andere Prozesse verschoben oder kopiert.
	\item {\bf Token-Objekte} werden zur Kommunikation verwendet und sind nicht veränderbar. Token-Objekte können zwischen den Prozessen verschoben werden, jedoch ist immer nur genau ein Prozess für ein jeweiliges Token-Objekt verantwortlich.
	\item {\bf Transiente Objekte} werden zu Kommunikationszwecken und zur Unterstützung der Algorithmen erstellt und aufbewahrt. Wenn ein transientes Objekt an einen anderen Prozess versendet wird, dann besitzen beide Prozess eine eigene Kopie des Objektes.
\end{enumerate}

Die folgenden Token-Objekte können identifiziert werden:
\begin{enumerate}
	\item Spielkaren: Zu beginn des Spieles wird für jede Spielkarte, welche in einen realen Ligretto-Spiel verwerdet würde, ein Spielkarten-Objekt erstellt.
\end{enumerate}

Folgende Zustands-Objekte können identifiziert werden:
\begin{enumerate}
	\item Handkarten-Stapel
	\item Ausgelegte Karten vor dem Mitspieler
	\item Stapel auf dem Spieltisch
	\item Mitspieler
	\item Überwachender Prozess
\end{enumerate}

Zur Kommunikation werden folgende Transiente Objekte erstellt:
\begin{enumerate}
	\item Start- und Stop-Signale des überwachenden Prozesses
	\item Stop-Signal eines Mitspielers
	\item Stapel mit gemischten Karten, welche vom überwachenden Prozess an die Mitspieler verteilt werden
\end{enumerate}


\subsection{Concurrency-Problem}

\subsubsection{Gleichzeitiges Legen einer Karte}
Es kommt vor, dass zwei Spieler gleichzeitig versuchen, die selbe Karte auf den selben Stapel zu legen. Wie im echten Spiel gilt auch in der Spielsimlulation: \textbf{First come first served}. Dabei werden alle RMI-Methoden mit dem \textit{synchronized}-Schlüsselwort synchronisiert.

\subsubsection{Spielstop währenddem eine Karte gelegt wird}
Wird das Spiel während einem Versuch, eine Karte auf den Stapel zu legen, beendet, darf die Karte gemäss Spielregeln nicht mehr gelegt werden. Dies kann erreicht werden, indem man die Erfolgsmeldung mit dem Spielstatus verknüpft. Ist das Spiel beim Aufruf der \textit{legeKarte}-Methode beendet, wird in jedem Fall \textit{false} zurückgegeben, beziehungsweise der legende Spieler interpretiert es als \textit{false}.

\subsection{Zuverlässigkeit}

\subsubsection{Verlorenes Netzwerkpaket}
Da wir uns für den Einsatz von RMI entschieden haben, brauchen wir uns um Paketverlust, Korrektheit der Daten, etc. keine Gedanken zu machen. RMI benutzt TCP zur Kommunikation und stellt somit die korrekte Übermittlung einer Nachricht sicher.

\subsubsection{Absturz eines Spielers}
Stürzt ein Spieler während des Spiels ab, so beeinträchtigt dies den Verlauf des Spiels nicht. Um eine lange Antwortzeit beim Legen einer Karte zu vermeiden, werden ihn seine Nachbarn für weitere Anfrageversuche ignorieren.

\subsubsection{Absturz des Servers}
Da der Server für die Übermittlung der Ligretto-Stop-Nachricht verantwortlich ist, hätte ein Serverabsturz zur Folge, dass alle Spieler spielen, bis sie ihren Ligrettostapel abgearbeitet haben, und dann beim Aufruf der \textit{ligrettoStop}-Methode erfahren, dass der Server keine Antwort mehr gibt. Der Spielausgang wäre damit verfälscht und nicht brauchbar.

\subsection{Skalierbarkeit}

Gemäss Spielregeln braucht es mindestens zwei Spieler für ein Ligrettospiel. Mit nur zwei Spielern ist die Chance jedoch hoch, dass das Spiel nicht zu Ende gespielt werden kann, da die Karten ungünstig liegen und damit ein Deadlock entsteht. Gegen oben existieren keine Grenzen. Da nur eine Kommunikation mit einer konstanten Anzahl Nachbarn erfolgt, bleibt die Netzwerkauslastung und der Rechenaufwand pro Spieler auch bei steigender Anzahl Gegner konstant. Je nach Implementation können Nachrichten, welche an alle Spieler gehen (\textit{spielStart} und \textit{spielEnde}) durch Grenzen der Netzwerkgeschwindigkeit etwas verspätet eintreffen.

%Bewertungsmassstab:
% Bei Ligretto berücksichtigen, dass die Anzahl Spieler steigen kann.
%    - Ist die Zahl der beteiligten Rechner definiert? Wieviele Rechner sind minimal nötig, wieviele können max. eingesetzt werden?
%    - Wie steigt die Leistung des Systems mit hinzugefügten Rechnern? (Bis zu 1 Zusatzpunkt für detaillierte Analyse)
